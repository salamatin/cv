%% The MIT License (MIT)
%%
%% Copyright (c) 2015 Daniil Belyakov
%%
%% Permission is hereby granted, free of charge, to any person obtaining a copy
%% of this software and associated documentation files (the "Software"), to deal
%% in the Software without restriction, including without limitation the rights
%% to use, copy, modify, merge, publish, distribute, sublicense, and/or sell
%% copies of the Software, and to permit persons to whom the Software is
%% furnished to do so, subject to the following conditions:
%%
%% The above copyright notice and this permission notice shall be included in all
%% copies or substantial portions of the Software.
%%
%% THE SOFTWARE IS PROVIDED "AS IS", WITHOUT WARRANTY OF ANY KIND, EXPRESS OR
%% IMPLIED, INCLUDING BUT NOT LIMITED TO THE WARRANTIES OF MERCHANTABILITY,
%% FITNESS FOR A PARTICULAR PURPOSE AND NONINFRINGEMENT. IN NO EVENT SHALL THE
%% AUTHORS OR COPYRIGHT HOLDERS BE LIABLE FOR ANY CLAIM, DAMAGES OR OTHER
%% LIABILITY, WHETHER IN AN ACTION OF CONTRACT, TORT OR OTHERWISE, ARISING FROM,
%% OUT OF OR IN CONNECTION WITH THE SOFTWARE OR THE USE OR OTHER DEALINGS IN THE
%% SOFTWARE.

% The font could be set to Windows-specific Calibri by using the 'calibri' option
\documentclass[]{cv}

% For hyperlinks
\usepackage{hyperref}

\makeatletter
\let\\\@raggedtwoe@savedcr% Restore original functionality of \\
\makeatother


% Set applicant's personal data for header
\name{Андрей Саламатин}
\address{\href{mailto:Salamatin.Andrey+job@gmail.com}{Salamatin.Andrey@gmail.com} \linebreak +375 (29) 985-96-83}
\contacts{\href{https://www.linkedin.com/in/salamatin/}{linkedin.com/in/salamatin}}

\begin{document}
\raggedright

	% Print the header
	\makeheader
	
	% Print the content
	\begin{cvsection}{Опыт работы}
		\begin{cvsubsection}{Senior Test Automation Engineer}{EPAM Systems}{апрель 2022 - март 2023}
		\end{cvsubsection}
		\begin{cvsubsection}{Test Automation Engineer}{}{март 2015 - апрель 2022}
			\begin{itemize}
				\item Разрабатывал и поддерживал автоматизированные тесты на 7+ аутсорс-проектах для англоязычных клиентов в области e-commerce, транспорта и здравоохранения.
				\item Покрывал автотестами UI и бэкенд (API, Azure Service Bus \& Event Hub) веб-приложений, валидацию XML, PDF, XLSX, e-mail рассылки, логирования и т.д.
				\item Настраивал и поддерживал CI и репортинг (Jenkins, в т.ч. pipelines, Azure DevOps, Report Portal).
				\item На 3-х долгосрочных проектах был AQA стрим-лидом, отвечая за планирование, распределение задач, код-ревью в команде из 2-3 автоматизаторов, коммуникацию с заказчиком и т.п.
				\newline
				\item Приняв устаревшие и на ~80\% нерабочие тесты плохого качества, в короткий срок перед важным релизом привёл их в порядок.
				\item В рамках инициативы Test as a Service отвечал за запуск автоматизированного тестирования для внутреннего приложения клиента: написал базовый TA-фреймворк, смок-тесты и стартовую документацию. Поддерживал работу 3-4 крауд-тестеров, рецензировал и интегрировал их код.
				\item Внедрил в тестовый фреймворк интеграцию с JIRA для экспорта Cucumber-сценариев и импорта результатов в Test Executions.
				\item Добавил во фреймворк возможность использовать для создания данных интеграционную шину вместо UI, что привело к сокращению времени выполнения до 300\% и большей стабильности тестов
				\item По проектной необходимости в кратчайший срок освоил JavaScript в достаточной cтепени, чтобы заменить профильного автоматизатора и покрывать тестами Node.js-микросервисы.
				\item Автоматизировал несколько бизнес-критичных процессов в SAP CDC, которые ранее должны были выполняться командой поддержки вручную несколько раз в день.
				\item Взяв ответственность за обучение англоязычного тестировщика со стороны заказчика со слабым знанием Java, помог ему достичь хорошего качества кода и сократить продолжительность код-ревью.			
				\item Осуществил многочисленные исправления в коде тестовых фреймворков.
				
			\end{itemize}
		\end{cvsubsection}
		\begin{cvsubsection}{Software Testing Engineer}{}{февраль 2013 - март 2015}
			\begin{itemize}
				\item Автоматизировал создание большого объёма тестовых данных с помощью Selenium Webdriver.
			\end{itemize}
		\end{cvsubsection}
	\end{cvsection}
	\begin{cvsection}{Образование}
		\begin{cvsubsection}{Карагандинский государственный технический университет}{}{2005-2010}
		«Вычислительная техника и программное обеспечение», бакалавриат
		\end{cvsubsection}
	\end{cvsection}	
	\begin{cvsection}{Языки и инструменты}
		\begin{cvsubsection}{}{}{}	
			\begin{itemize}
				\item Java, SQL
				\item Selenium for Java, Serenity BDD, Apickli, Selenide. TestNG, JUnit. Maven, Gradle. Jenkins Pipelines, Azure DevOps, Report Portal
				\item Oracte ATG, SAP C4C, SAP CDC
			\end{itemize}
		\end{cvsubsection}
	\end{cvsection}
	
\end{document}

